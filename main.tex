% ------------------------------------------------------------------------
%%%%%%%%%%%%%%%%%%%%%%%%%%%%%%%%%%%%%%%%%%%%%%%%%%%%%%%%%%%%%%%%%%%%%%%%%%
% ------------------------------------------------------------------------

% TRABALHO DE CONCLUSÃO DE CURSO - QUÍMICA INDUSTRIAL

% ------------------------------------------------------------------------
%%%%%%%%%%%%%%%%%%%%%%%%%%%%%%%%%%%%%%%%%%%%%%%%%%%%%%%%%%%%%%%%%%%%%%%%%%
% ------------------------------------------------------------------------

\documentclass[
	% -- opções da classe memoir --
	12pt,				% tamanho da fonte
	openright,			% capítulos começam em pág ímpar (insere página vazia caso preciso)
	oneside,			% para impressão em recto e verso. Oposto a oneside
	a4paper,			% tamanho do papel. 
	% -- opções da classe abntex2 --
	%chapter=TITLE,		% títulos de capítulos convertidos em letras maiúsculas
	%section=TITLE,		% títulos de seções convertidos em letras maiúsculas
	%subsection=TITLE,	% títulos de subseções convertidos em letras maiúsculas
	%subsubsection=TITLE,% títulos de subsubseções convertidos em letras maiúsculas
	% -- opções do pacote babel --
	english,			% idioma adicional para hifenização
	french,				% idioma adicional para hifenização
	spanish,			% idioma adicional para hifenização
	brazil				% o último idioma é o principal do documento
	]{abntex2}

% ---
% Pacotes básicos 
% ---
\usepackage{lmodern}			% Usa a fonte Latin Modern			
\usepackage[T1]{fontenc}		% Selecao de codigos de fonte.
\usepackage[utf8]{inputenc}		% Codificacao do documento (conversão automática dos acentos)
\usepackage{indentfirst}		% Indenta o primeiro parágrafo de cada seção.
\usepackage{color}				% Controle das cores
\usepackage{graphicx}			% Inclusão de gráficos
\usepackage{microtype} 			% para melhorias de justificação
\usepackage{geometry}           % ajuste personalizado de margens 
% ---
		
% ---
% Pacotes adicionais, usados apenas no âmbito do Modelo Canônico do abnteX2
% ---
\usepackage{lipsum}				% para geração de dummy text
% ---

% ---
% Pacotes de citações
% ---
\usepackage[brazilian,hyperpageref]{backref}	 % Paginas com as citações na bibl
\usepackage[alf, abnt-repeated-title-omit=yes,abnt-emphasize=bf,abnt-etal-list=0]{abntex2cite}	% Citações padrão ABNT
% ---
% ---
% Pacotes para importação de tabelas do Google Sheets
% ---
\usepackage{booktabs, multirow} % for borders and merged ranges
\usepackage{soul}% for underlines
\usepackage[table]{xcolor} % for cell colors
\usepackage{changepage,threeparttable} % for wide tables
\usepackage[export]{adjustbox}
\usepackage{subcaption} % legendas de figuras múltiplas
\usepackage{float} % posicionamento de figuras e tabelas
% --- 
% CONFIGURAÇÕES DE PACOTES
% --- 

% ---
% Configurações do pacote backref
% Usado sem a opção hyperpageref de backref
\renewcommand{\backrefpagesname}{Citado na(s) página(s):~}
% Texto padrão antes do número das páginas
\renewcommand{\backref}{}
% Define os textos da citação
\renewcommand*{\backrefalt}[4]{
	\ifcase #1 %
		Nenhuma citação no texto.%
	\or
		Citado na página #2.%
	\else
		Citado #1 vezes nas páginas #2.%
	\fi}%
% ---

% ---
% Informações de dados para CAPA e FOLHA DE ROSTO
% ---
\titulo{Concentração de ozônio troposférico:\\ 
fatores antrópicos e meteorológicos} % Subtítulo

\autor{Antonio Guerreiro Silva Serrano}

\local{Diadema}

\data{2023}

\orientador{Mauro Aquiles La Scalea}

% \coorientador{Equipe \abnTeX}

\instituicao{Universidade Federal de São Paulo \\
Campus Diadema} % Nome do campus

\tipotrabalho{Trabalho de Conclusão de Curso (Graduação em Química Industrial)}

% ---


% ---
% Configurações de aparência do PDF final

% alterando o aspecto da cor azul
\definecolor{blue}{RGB}{41,5,195}

% informações do PDF
\makeatletter
\hypersetup{
     	%pagebackref=true,
		pdftitle={\@title}, 
		pdfauthor={\@author},
    	pdfsubject={\imprimirpreambulo},
	    pdfcreator={LaTeX with abnTeX2},
		pdfkeywords={abnt}{latex}{abntex}{abntex2}{trabalho acadêmico}, 
		colorlinks=true,       		% false: boxed links; true: colored links
    	linkcolor=blue,          	% color of internal links
    	citecolor=blue,        		% color of links to bibliography
    	filecolor=magenta,      		% color of file links
		urlcolor=blue,
		bookmarksdepth=4
}
\makeatother
% --- 

% ---
% Posiciona figuras e tabelas no topo da página quando adicionadas sozinhas
% em um página em branco. Ver https://github.com/abntex/abntex2/issues/170
\makeatletter
\setlength{\@fptop}{5pt} % Set distance from top of page to first float
\makeatother
% ---

% ---
% Possibilita criação de Quadros e Lista de quadros.
% Ver https://github.com/abntex/abntex2/issues/176
%
% \newcommand{\quadroname}{Quadro}
% \newcommand{\listofquadrosname}{Lista de quadros}

% \newfloat[chapter]{quadro}{loq}{\quadroname}
% \newlistof{listofquadros}{loq}{\listofquadrosname}
% \newlistentry{quadro}{loq}{0}

% % configurações para atender às regras da ABNT
% \setfloatadjustment{quadro}{\centering}
% \counterwithout{quadro}{chapter}
% \renewcommand{\cftquadroname}{\quadroname\space} 
% \renewcommand*{\cftquadroaftersnum}{\hfill--\hfill}

% \setfloatlocations{quadro}{hbtp} % Ver https://github.com/abntex/abntex2/issues/176
% % ---

% --- 
% Espaçamentos entre linhas e parágrafos 
% --- 

% O tamanho do parágrafo é dado por:
\setlength{\parindent}{1.3cm}

% Controle do espaçamento entre um parágrafo e outro:
\setlength{\parskip}{0.2cm}  % tente também \onelineskip

% ---
% compila o indice
% ---
\makeindex
% ---

% ----
% Início do documento
% ----
\begin{document}

% Seleciona o idioma do documento (conforme pacotes do babel)
%\selectlanguage{english}
\selectlanguage{brazil}

% Retira espaço extra obsoleto entre as frases.
\frenchspacing 

% ---
% Comandos inseridos de arquivo externo 
% ---
\include{abntex2-modelo-include-comandos}
% ---

%%%%%%%%%%%%%%%%%%%%%%%%%%%%%%%%%%%%%%%%%%%%%%%%%%%%%%%%%%%%%%


%  E L E M E N T O S   P R É - T E X T U A I S  (OBRIGATÓRIO)


%%%%%%%%%%%%%%%%%%%%%%%%%%%%%%%%%%%%%%%%%%%%%%%%%%%%%%%%%%%%%%
\pretextual

%%%%%%%%%%%%%%%%%%%%%%%%%%%%%%%%%%%%%%%%%%%%%%%%%%%%%%%%%%%%
% C A P A (OBRIGATÓRIO)
%%%%%%%%%%%%%%%%%%%%%%%%%%%%%%%%%%%%%%%%%%%%%%%%%%%%%%%%%%%
% ----------------------------------------------------------------------- 
% Capa personalizada com o logo institucional e o brasão da República 
% ----------------------------------------------------------------------- 
\newpage
\thispagestyle{empty}
\newgeometry{left=0cm,right=0cm,top=3cm,bottom=2cm}
%------------------------------------------------------------------------
% Cabeçalho
%------------------------------------------------------------------------
\begin{center}
\begin{minipage}{.15\textwidth}  
\includegraphics[width=\textwidth]{./images/brasao_republica} %Imagem 1
\end{minipage}%
\begin{minipage}{.5\textwidth}
\centering
\textbf{\MakeUppercase{\imprimirinstituicao}} % Texto
\end{minipage}%
\begin{minipage}{.2\textwidth}
\includegraphics[width=\textwidth]{./images/logo_unifesp} %Imagem 2
\end{minipage}% 
\end{center}
%-------------------------------------------------------------------------
% Informaçoes do trabalho
%-------------------------------------------------------------------------
\vspace{3cm} % Espaçamento cabeçalho/nome do autor

\begin{center}

\large{\MakeUppercase{\imprimirautor}} % Nome do discente

\vspace{5cm} % Espaçamento nome do autor/título

\textbf{\large{\MakeUppercase{\imprimirtitulo}}} % Título com/sem Subtítulo

\vspace{7.5cm} % Espaçamento título/local e data

\large{\MakeUppercase{\imprimirlocal}}\break % Cidade/Estado
\imprimirdata % Ano

\end{center}
%-------------------------------------------------------------------------

\restoregeometry




% ---

%%%%%%%%%%%%%%%%%%%%%%%%%%%%%%%%%%%%%%%%%%%%%%%%%%%%%%%%%%%%
% F O L H A  D E  R O S T O  (OBRIGATÓRIO)
%%%%%%%%%%%%%%%%%%%%%%%%%%%%%%%%%%%%%%%%%%%%%%%%%%%%%%%%%%
% (o * indica que haverá a ficha bibliográfica)
% ---
\imprimirfolhaderosto*
% ---

%%%%%%%%%%%%%%%%%%%%%%%%%%%%%%%%%%%%%%%%%%%%%%%%%%%%%%%%%%%%
% F I C H A   C A T A L O G R Á F I A  (OBRIGATÓRIO)
%%%%%%%%%%%%%%%%%%%%%%%%%%%%%%%%%%%%%%%%%%%%%%%%%%%%%%%%%%%%

% Isto é um exemplo de Ficha Catalográfica, ou ``Dados internacionais de
% catalogação-na-publicação''. Você pode utilizar este modelo como referência. 
% Porém, provavelmente a biblioteca da sua universidade lhe fornecerá um PDF
% com a ficha catalográfica definitiva após a defesa do trabalho. Quando estiver
% com o documento, salve-o como PDF no diretório do seu projeto e substitua todo
% o conteúdo de implementação deste arquivo pelo comando abaixo:
%
% \begin{fichacatalografica}
%     \includepdf{fig_ficha_catalografica.pdf}
% \end{fichacatalografica}

%----------------------------------------------------------------------
%Definição de variáveis específicas para a ficha catalográfica
%----------------------------------------------------------------------
\def \autor2 {Serrano, Antonio Guerreiro Silva}
\def \titulo2 {Concentração de ozônio troposférico: fatores antrópicos e meteorológicos}
\def \instituicaoumalinha {Universidade Federal de São Paulo -- Campus Diadema}
%----------------------------------------------------------------------

\begin{fichacatalografica}
	\sffamily
	\vspace*{\fill}					% Posição vertical
	\begin{center}					% Minipage Centralizado
	\fbox{\begin{minipage}[c][7.5cm]{12.5cm}		% Largura
	\small
	\autor2
		
	\hspace{0.5cm} \titulo2  / \imprimirautor. --
	\imprimirlocal, \imprimirdata .
	
	\hspace{0.5cm} \thelastpage \, f.\\
	
	\hspace{0.5cm} \imprimirorientadorRotulo~\imprimirorientador\\
	
	\hspace{0.5cm}
	 \parbox[t]{\textwidth}{\imprimirtipotrabalho~--~ \\ 
	 \instituicaoumalinha, \imprimirdata.}\\~\\
	
	\hspace{0.5cm}
		1. Ozônio. %palavra-chave 1
		2. Troposfera. %palavra-chave 2
		2. Antropogênico. %palavra-chave 3
		2. Meteorologia. %palavra-chave 4
		% I. Orientador.
		% II. Universidade xxx.
		% III. Faculdade de xxx.
		% IV. Título 
		I. Título			
	\end{minipage}}
	\end{center}
\end{fichacatalografica}
% ---

%%%%%%%%%%%%%%%%%%%%%%%%%%%%%%%%%%%%%%%%%%%%%%%%%%%%%%%%%
% E R R A T A (OPCIONAL)
%%%%%%%%%%%%%%%%%%%%%%%%%%%%%%%%%%%%%%%%%%%%%%%%%%%%%%%%%
% \begin{errata}
% Elemento opcional da \citeonline[4.2.1.2]{NBR14724:2011}. Exemplo:

% \vspace{\onelineskip}

% FERRIGNO, C. R. A. \textbf{Tratamento de neoplasias ósseas apendiculares com
% reimplantação de enxerto ósseo autólogo autoclavado associado ao plasma
% rico em plaquetas}: estudo crítico na cirurgia de preservação de membro em
% cães. 2011. 128 f. Tese (Livre-Docência) - Faculdade de Medicina Veterinária e
% Zootecnia, Universidade de São Paulo, São Paulo, 2011.

% \begin{table}[htb]
% \center
% \footnotesize
% \begin{tabular}{|p{1.4cm}|p{1cm}|p{3cm}|p{3cm}|}
%   \hline
%    \textbf{Folha} & \textbf{Linha}  & \textbf{Onde se lê}  & \textbf{Leia-se}  \\
%     \hline
%     1 & 10 & auto-conclavo & autoconclavo\\
%    \hline
% \end{tabular}
% \end{table}

% \end{errata}
% ---

%%%%%%%%%%%%%%%%%%%%%%%%%%%%%%%%%%%%%%%%%%%%%%%%%%%%%%%%%
% F O L H A   D E   A P R O V A Ç Ã O  (OBRIGATÓRIO)
%%%%%%%%%%%%%%%%%%%%%%%%%%%%%%%%%%%%%%%%%%%%%%%%%%%%%%%%%

% Isto é um exemplo de Folha de aprovação, elemento obrigatório da NBR
% 14724/2011 (seção 4.2.1.3). Você pode utilizar este modelo até a aprovação
% do trabalho. Após isso, substitua todo o conteúdo deste arquivo por uma
% imagem da página assinada pela banca com o comando abaixo:
%
% \begin{folhadeaprovacao}
% \includepdf{folhadeaprovacao_final.pdf}
% \end{folhadeaprovacao}

\begin{folhadeaprovacao}

  \begin{center}
    {\ABNTEXchapterfont\large\imprimirautor}

    \vspace*{\fill}\vspace*{\fill}
    \begin{center}
      \ABNTEXchapterfont\bfseries\Large\imprimirtitulo
    \end{center}
    \vspace*{\fill}
    
    \hspace{.45\textwidth}
    \begin{minipage}{.5\textwidth}
        \imprimirpreambulo
    \end{minipage}%
    \vspace*{\fill}
   \end{center}
        
   Trabalho aprovado. \imprimirlocal, xx de xxxxxx de \imprimirdata: %nome do mês por extenso

   \assinatura{\textbf{\imprimirorientador} \\ Orientador} 
   \assinatura{\textbf{Professor} \\ Convidado 1}
   \assinatura{\textbf{Professor} \\ Convidado 2}
   \assinatura{\textbf{Professor} \\ Convidado 3}
   %\assinatura{\textbf{Professor} \\ Convidado 4}
      
   \begin{center}
    \vspace*{0.5cm}
    {\large\imprimirlocal}
    \par
    {\large\imprimirdata}
    \vspace*{1cm}
  \end{center}
  
\end{folhadeaprovacao}
% ---

%%%%%%%%%%%%%%%%%%%%%%%%%%%%%%%%%%%%%%%%%%%%%%%%%%%%%%%%%%%
% D E D I C A T O R I A  (OPCIONAL)
%%%%%%%%%%%%%%%%%%%%%%%%%%%%%%%%%%%%%%%%%%%%%%%%%%%%%%%%%%%
% \begin{dedicatoria}
%    \vspace*{\fill}
%    \centering
%    \noindent
%    \textit{ Este trabalho é dedicado às crianças adultas que,\\
%    quando pequenas, sonharam em se tornar cientistas.} \vspace*{\fill}
% \end{dedicatoria}
% ---

%%%%%%%%%%%%%%%%%%%%%%%%%%%%%%%%%%%%%%%%%%%%%%%%%%%%%%%%%%%%
% A G R A D E C I M E N T O S  (OPCIONAL)
%%%%%%%%%%%%%%%%%%%%%%%%%%%%%%%%%%%%%%%%%%%%%%%%%%%%%%%%%%%%
% \begin{agradecimentos}
% Os agradecimentos principais são direcionados à Gerald Weber, Miguel Frasson,
% Leslie H. Watter, Bruno Parente Lima, Flávio de Vasconcellos Corrêa, Otavio Real
% Salvador, Renato Machnievscz\footnote{Os nomes dos integrantes do primeiro
% projeto abn\TeX\ foram extraídos de
% \url{http://codigolivre.org.br/projects/abntex/}} e todos aqueles que
% contribuíram para que a produção de trabalhos acadêmicos conforme
% as normas ABNT com \LaTeX\ fosse possível.

% Agradecimentos especiais são direcionados ao Centro de Pesquisa em Arquitetura
% da Informação\footnote{\url{http://www.cpai.unb.br/}} da Universidade de
% Brasília (CPAI), ao grupo de usuários
% \emph{latex-br}\footnote{\url{http://groups.google.com/group/latex-br}} e aos
% novos voluntários do grupo
% \emph{\abnTeX}\footnote{\url{http://groups.google.com/group/abntex2} e
% \url{http://www.abntex.net.br/}}~que contribuíram e que ainda
% contribuirão para a evolução do \abnTeX.

% \end{agradecimentos}
% ---

%%%%%%%%%%%%%%%%%%%%%%%%%%%%%%%%%%%%%%%%%%%%%%%%%%%%%%%%%%%%
% E P I G R A F E (OPCIONAL E SEM TÍTULO)
%%%%%%%%%%%%%%%%%%%%%%%%%%%%%%%%%%%%%%%%%%%%%%%%%%%%%%%%%%%%
% \begin{epigrafe}
%     \vspace*{\fill}
% 	\begin{flushright}
% 		\textit{``Não vos amoldeis às estruturas deste mundo, \\
% 		mas transformai-vos pela renovação da mente, \\
% 		a fim de distinguir qual é a vontade de Deus: \\
% 		o que é bom, o que Lhe é agradável, o que é perfeito.\\
% 		(Bíblia Sagrada, Romanos 12, 2)}
% 	\end{flushright}
% \end{epigrafe}
% ---

%%%%%%%%%%%%%%%%%%%%%%%%%%%%%%%%%%%%%%%%%%%%%%%%%%%%%%%%%%%%
% R E S U M O    E M    P O R T U G U Ê S  (OBRIGATÓRIO)
%%%%%%%%%%%%%%%%%%%%%%%%%%%%%%%%%%%%%%%%%%%%%%%%%%%%%%%%%%%%
%---------------------------------------------
% Resumo em português
%---------------------------------------------
\setlength{\absparsep}{18pt} % ajusta o espaçamento dos parágrafos do resumo
\begin{resumo}
 O resumo deve ressaltar o
 objetivo, o método, os resultados e as conclusões do documento. A ordem e a extensão
 destes itens dependem do tipo de resumo (informativo ou indicativo) e do
 tratamento que cada item recebe no documento original. O resumo deve ser
 precedido da referência do documento, com exceção do resumo inserido no
 próprio documento. (\ldots) As palavras-chave devem figurar logo abaixo do
 resumo, antecedidas da expressão Palavras-chave:, separadas entre si por
 ponto e finalizadas também por ponto.

 \textbf{Palavras-chave}: latex. abntex. editoração de texto.
\end{resumo}
% ---

%%%%%%%%%%%%%%%%%%%%%%%%%%%%%%%%%%%%%%%%%%%%%%%%%%%%%%%%%%%%
% A B S T R A C T  (OBRIGATÓRIO)
%%%%%%%%%%%%%%%%%%%%%%%%%%%%%%%%%%%%%%%%%%%%%%%%%%%%%%%%%%%%
%---------------------------------------------
% Resumo em inglês
%---------------------------------------------
\begin{resumo}[Abstract]
    \begin{otherlanguage*}{english}
      This is the english abstract.
   
      \vspace{\onelineskip}
    
      \noindent 
      \textbf{Keywords}: latex. abntex. text editoration.
    \end{otherlanguage*}
   \end{resumo}
% ---

%%%%%%%%%%%%%%%%%%%%%%%%%%%%%%%%%%%%%%%%%%%%%%%%%%%%%%%%%%%%
% L I S T A   D E   I L U S T R A Ç Õ E S   (OBRIGATÓRIO)
%%%%%%%%%%%%%%%%%%%%%%%%%%%%%%%%%%%%%%%%%%%%%%%%%%%%%%%%%%%%
\pdfbookmark[0]{\listfigurename}{lof}
\listoffigures*
\cleardoublepage
% ---

%%%%%%%%%%%%%%%%%%%%%%%%%%%%%%%%%%%%%%%%%%%%%%%%%%%%%%%%%%%%
% L I S T A   D E   Q U A D R O S  (OPCIONAL)
%%%%%%%%%%%%%%%%%%%%%%%%%%%%%%%%%%%%%%%%%%%%%%%%%%%%%%%%%%%%
% \pdfbookmark[0]{\listofquadrosname}{loq}
% \listofquadros*
% \cleardoublepage
% ---

%%%%%%%%%%%%%%%%%%%%%%%%%%%%%%%%%%%%%%%%%%%%%%%%%%%%%%%%%%%%
% L I S T A   D E   C Ó D I G O S (OPCIONAL)
%%%%%%%%%%%%%%%%%%%%%%%%%%%%%%%%%%%%%%%%%%%%%%%%%%%%%%%%%%%%
% \pdfbookmark[0]{\lstlistlistingname}{lol}
% \begin{KeepFromToc}
% \lstlistoflistings
% \end{KeepFromToc}
% \cleardoublepage
% ---

%%%%%%%%%%%%%%%%%%%%%%%%%%%%%%%%%%%%%%%%%%%%%%%%%%%%%%%%%%%%
% L I S T A   D E   T A B E L A S   (OBRIGATÓRIO)
%%%%%%%%%%%%%%%%%%%%%%%%%%%%%%%%%%%%%%%%%%%%%%%%%%%%%%%%%%%%
\pdfbookmark[0]{\listtablename}{lot}
\listoftables*
\cleardoublepage
% ---

%%%%%%%%%%%%%%%%%%%%%%%%%%%%%%%%%%%%%%%%%%%%%%%%%%%%%%%%%%%%
% L I S T A   D E  A B R E V I A T U R A S  E   S I G L A S 
% (OPCIONAL)
%%%%%%%%%%%%%%%%%%%%%%%%%%%%%%%%%%%%%%%%%%%%%%%%%%%%%%%%%%%%
\begin{siglas}
    \item[ABNT] Associação Brasileira de Normas Técnicas
    \item[abnTeX] ABsurdas Normas para TeX
  \end{siglas}
% ---

%%%%%%%%%%%%%%%%%%%%%%%%%%%%%%%%%%%%%%%%%%%%%%%%%%%%%%%%%%%%
% L I S T A   D E  S Í M B O L O S   (OPCIONAL)
%%%%%%%%%%%%%%%%%%%%%%%%%%%%%%%%%%%%%%%%%%%%%%%%%%%%%%%%%%%%
\begin{simbolos}
    \item[$ \Gamma $] Letra grega Gama
    \item[$ \Lambda $] Lambda
    \item[$ \zeta $] Letra grega minúscula zeta
    \item[$ \in $] Pertence
  \end{simbolos}
% ---

%%%%%%%%%%%%%%%%%%%%%%%%%%%%%%%%%%%%%%%%%%%%%%%%%%%%%%%%%%%%
%  S U M Á R I O  (OBRIGATÓRIO)
%%%%%%%%%%%%%%%%%%%%%%%%%%%%%%%%%%%%%%%%%%%%%%%%%%%%%%%%%%%%
\pdfbookmark[0]{\contentsname}{toc}
\tableofcontents*
\cleardoublepage
% ---

%%%%%%%%%%%%%%%%%%%%%%%%%%%%%%%%%%%%%%%%%%%%%%%%%%%%%%%%%%%%


%  E L E M E N T O S   T E X T U A I S  (OBRIGATÓRIO)


%%%%%%%%%%%%%%%%%%%%%%%%%%%%%%%%%%%%%%%%%%%%%%%%%%%%%%%%%%%%
\textual

% ----------------------------------------------------------
% INTRODUÇÃO 
% ----------------------------------------------------------

% ----------------------------------------------------------
% Introdução (exemplo de capítulo sem numeração, mas presente no Sumário)
% ----------------------------------------------------------
\chapter{Introdução}
% ----------------------------------------------------------

Este documento e seu código-fonte são exemplos de referência de uso da classe
\textsf{abntex2} e do pacote \textsf{abntex2cite}. O documento 
exemplifica a elaboração de trabalho acadêmico (tese, dissertação e outros do
gênero) produzido conforme a ABNT NBR 14724:2011 \emph{Informação e documentação
- Trabalhos acadêmicos - Apresentação}.

A expressão ``Modelo Canônico'' é utilizada para indicar que \abnTeX\ não é
modelo específico de nenhuma universidade ou instituição, mas que implementa tão
somente os requisitos das normas da ABNT. Uma lista completa das normas
observadas pelo \abnTeX\ é apresentada em \citeonline{abntex2classe}.

Sinta-se convidado a participar do projeto \abnTeX! Acesse o site do projeto em
\url{http://www.abntex.net.br/}. Também fique livre para conhecer,
estudar, alterar e redistribuir o trabalho do \abnTeX, desde que os arquivos
modificados tenham seus nomes alterados e que os créditos sejam dados aos
autores originais, nos termos da ``The \LaTeX\ Project Public
License''\footnote{\url{http://www.latex-project.org/lppl.txt}}.

Encorajamos que sejam realizadas customizações específicas deste exemplo para
universidades e outras \cite{sazonalidade} instituições --- como capas, folha de aprovação, etc.
Porém, recomendamos que ao invés de se alterar diretamente os arquivos do
\abnTeX, distribua-se arquivos com as respectivas customizações \cite{cetesb-poluicao-veicular}.
Isso permite que futuras versões do \abnTeX~não se tornem automaticamente
incompatíveis com as customizações promovidas. Consulte
\citeonline{abntex2-wiki-como-customizar} para mais informações\cite{sazonalidade}.

Este documento deve ser utilizado como complemento dos manuais do \abnTeX\ 
\cite{abntex2classe,abntex2cite,abntex2cite-alf} e da classe \textsf{memoir}
\cite{memoir}. 



% ----------------------------------------------------------
% PARTE
% ----------------------------------------------------------
\part{Preparação da pesquisa}
% ----------------------------------------------------------

% ---
% primeiro capítulo 
% ---
\chapter{Contextualização e motivação}
% ---

\lipsum[21-22]

% ---
% segundo capítulo 
% ---
\chapter{Objetivos gerais e específicos}
% ---

\lipsum[24-25]

% ----------------------------------------------------------
% PARTE
% ----------------------------------------------------------
\part{Referenciais teóricos}
% ----------------------------------------------------------

% primeiro capitulo 
% ---
\chapter{Poluição atmosférica em centros urbanos}
% ---

\lipsum[21-22]

% ---
% primeiro subcapítulo
% ---
\section{Histórico da poluição atmosférica na RMSP}
% ---

\lipsum[21-22]

% ---
% segundo subcapítulo
% ---
\section{Ozônio troposférico}
% ---

\lipsum[24-25]

% ---
% terceiro subcapítulo
% ---
\section{Formação e transporte de ozônio na troposfera}
% ---

\lipsum[24-25]

% ---
% quarto subcapítulo
% ---
\section{Redução da poluição atmosférica na RMSP}
% ---

\lipsum[24-25]

% ---
% quinto subcapítulo
% ---
\section{Técnicas analíticas de medição}
% ---

\subsection{O$_{3}$}

\lipsum[24-25]

\subsection{NO}

\lipsum[24-25]

\subsection{NO$_{2}$}

\lipsum[24-25]

% ----------------------------------------------------------
% PARTE
% ----------------------------------------------------------
\part{Origem dos dados}
% ----------------------------------------------------------

%---
% P R I M E  I R O  C A P Í T U L O 
% ---
\chapter{Bancos de dados}
% ---


%%%%%%%%%%%%%%%%%%%%%%%%%%%%%%%%%%%%%%%%%%%%%%%%%%%%%%%%%%%%%%%%%%%%%%%%%%%%%%%%
%Texto 

\lipsum[100]

%%%%%%%%%%%%%%%%%%%%%%%%%%%%%%%%%%%%%%%%%%%%%%%%%%%%%%%%%%%%%%%%%%%%%%%%%%%%%%%% 


% ---
% PRIMEIRO SUBCAPÍTULO
% ---
\section{QUALAR}
% ---


%%%%%%%%%%%%%%%%%%%%%%%%%%%%%%%%%%%%%%%%%%%%%%%%%%%%%%%%%%%%%%%%%%%%%%%%%%%%%%%%
%Texto 

\lipsum[100]

%%%%%%%%%%%%%%%%%%%%%%%%%%%%%%%%%%%%%%%%%%%%%%%%%%%%%%%%%%%%%%%%%%%%%%%%%%%%%%%% 


% ---
% SEGUNDO SUBCAPÍTULO
% ---
\section{INMET}
% ---


%%%%%%%%%%%%%%%%%%%%%%%%%%%%%%%%%%%%%%%%%%%%%%%%%%%%%%%%%%%%%%%%%%%%%%%%%%%%%%%%
%Texto 

\lipsum[100]

%%%%%%%%%%%%%%%%%%%%%%%%%%%%%%%%%%%%%%%%%%%%%%%%%%%%%%%%%%%%%%%%%%%%%%%%%%%%%%%% 

% ---
% TERCEIRO SUBCAPÍTULO
% ---
\section{IAG-USP}
% ---


%%%%%%%%%%%%%%%%%%%%%%%%%%%%%%%%%%%%%%%%%%%%%%%%%%%%%%%%%%%%%%%%%%%%%%%%%%%%%%%%
%Texto 

\lipsum[100]

%%%%%%%%%%%%%%%%%%%%%%%%%%%%%%%%%%%%%%%%%%%%%%%%%%%%%%%%%%%%%%%%%%%%%%%%%%%%%%%% 


%---
% S E G  U N D O  C A P Í T U L O  
% ---
\chapter{Caracterização da área e período de estudo}
% ---

% ---
% PRIMEIRO SUBCAPÍTULO
% ---
\section{Área de estudo}
% ---


%%%%%%%%%%%%%%%%%%%%%%%%%%%%%%%%%%%%%%%%%%%%%%%%%%%%%%%%%%%%%%%%%%%%%%%%%%%%%%%%
%Texto 

\lipsum[100]

%%%%%%%%%%%%%%%%%%%%%%%%%%%%%%%%%%%%%%%%%%%%%%%%%%%%%%%%%%%%%%%%%%%%%%%%%%%%%%%% 


% ---
% SEGUNDO SUBCAPÍTULO
% ---
\section{Período de estudo}
% ---


%%%%%%%%%%%%%%%%%%%%%%%%%%%%%%%%%%%%%%%%%%%%%%%%%%%%%%%%%%%%%%%%%%%%%%%%%%%%%%%%
%Texto 

\lipsum[100]

%%%%%%%%%%%%%%%%%%%%%%%%%%%%%%%%%%%%%%%%%%%%%%%%%%%%%%%%%%%%%%%%%%%%%%%%%%%%%%%% 

% ----------------------------------------------------------
% PARTE
% ----------------------------------------------------------
\part{Materiais, Métodos e Conceitos}
% ----------------------------------------------------------

%---
% P R I M E  I R O  C A P Í T U L O 
% ---
\chapter{Materiais}
%---


%%%%%%%%%%%%%%%%%%%%%%%%%%%%%%%%%%%%%%%%%%%%%%%%%%%%%%%%%%%%%%%%%%%%%%%%%%%%%%%%
%Texto 

\lipsum[100]

%%%%%%%%%%%%%%%%%%%%%%%%%%%%%%%%%%%%%%%%%%%%%%%%%%%%%%%%%%%%%%%%%%%%%%%%%%%%%%%% 


%---
% S E G U N D O  C A P Í T U L O  
% ---
\chapter{Pré-tratamento dos dados}


%%%%%%%%%%%%%%%%%%%%%%%%%%%%%%%%%%%%%%%%%%%%%%%%%%%%%%%%%%%%%%%%%%%%%%%%%%%%%%%%
%Texto 

\lipsum[100]

%%%%%%%%%%%%%%%%%%%%%%%%%%%%%%%%%%%%%%%%%%%%%%%%%%%%%%%%%%%%%%%%%%%%%%%%%%%%%%%% 


% ---
% PRIMEIRO SUBCAPÍTULO
% ---
\section{Coleta dos dados}
% ---


%%%%%%%%%%%%%%%%%%%%%%%%%%%%%%%%%%%%%%%%%%%%%%%%%%%%%%%%%%%%%%%%%%%%%%%%%%%%%%%%
%Texto 

\lipsum[100]

%%%%%%%%%%%%%%%%%%%%%%%%%%%%%%%%%%%%%%%%%%%%%%%%%%%%%%%%%%%%%%%%%%%%%%%%%%%%%%%% 


% ---
% SEGUNDO SUBCAPÍTULO
% ---
\section{Preparação dos dados}
% ---


%%%%%%%%%%%%%%%%%%%%%%%%%%%%%%%%%%%%%%%%%%%%%%%%%%%%%%%%%%%%%%%%%%%%%%%%%%%%%%%%
%Texto 

\lipsum[100]

%%%%%%%%%%%%%%%%%%%%%%%%%%%%%%%%%%%%%%%%%%%%%%%%%%%%%%%%%%%%%%%%%%%%%%%%%%%%%%%% 


%---
% T E R C E I R O  C A P Í T U L O  
% ---
\chapter{Regressão Linear}
%---


%%%%%%%%%%%%%%%%%%%%%%%%%%%%%%%%%%%%%%%%%%%%%%%%%%%%%%%%%%%%%%%%%%%%%%%%%%%%%%%%
%Texto 

\lipsum[100]

%%%%%%%%%%%%%%%%%%%%%%%%%%%%%%%%%%%%%%%%%%%%%%%%%%%%%%%%%%%%%%%%%%%%%%%%%%%%%%%% 


% ---
% PRIMEIRO SUBCAPÍTULO
% ---
\section{Coeficiente de correlação de Pearson}
% ---


%%%%%%%%%%%%%%%%%%%%%%%%%%%%%%%%%%%%%%%%%%%%%%%%%%%%%%%%%%%%%%%%%%%%%%%%%%%%%%%%
%Texto 

\lipsum[100]

%%%%%%%%%%%%%%%%%%%%%%%%%%%%%%%%%%%%%%%%%%%%%%%%%%%%%%%%%%%%%%%%%%%%%%%%%%%%%%%% 


%---
% Q U A R T O  C A P Í T U L O  
% ---
\chapter{Análise de Séries Temporais}
% ---


%%%%%%%%%%%%%%%%%%%%%%%%%%%%%%%%%%%%%%%%%%%%%%%%%%%%%%%%%%%%%%%%%%%%%%%%%%%%%%%%
%Texto 

\lipsum[100]

%%%%%%%%%%%%%%%%%%%%%%%%%%%%%%%%%%%%%%%%%%%%%%%%%%%%%%%%%%%%%%%%%%%%%%%%%%%%%%%% 


% ---
% PRIMEIRO SUBCAPÍTULO
% ---
\section{Decomposição da série}
% ---


%%%%%%%%%%%%%%%%%%%%%%%%%%%%%%%%%%%%%%%%%%%%%%%%%%%%%%%%%%%%%%%%%%%%%%%%%%%%%%%%
%Texto 

\lipsum[100]

%%%%%%%%%%%%%%%%%%%%%%%%%%%%%%%%%%%%%%%%%%%%%%%%%%%%%%%%%%%%%%%%%%%%%%%%%%%%%%%% 


% ---
% SEGUNDO SUBCAPÍTULO
% ---
\section{Estacionaridade da série}
% ---


%%%%%%%%%%%%%%%%%%%%%%%%%%%%%%%%%%%%%%%%%%%%%%%%%%%%%%%%%%%%%%%%%%%%%%%%%%%%%%%%
%Texto 

\lipsum[100]

%%%%%%%%%%%%%%%%%%%%%%%%%%%%%%%%%%%%%%%%%%%%%%%%%%%%%%%%%%%%%%%%%%%%%%%%%%%%%%%% 


% ---
% TERCEIRO SUBCAPÍTULO
% ---
\section{Correlograma}
% ---


%%%%%%%%%%%%%%%%%%%%%%%%%%%%%%%%%%%%%%%%%%%%%%%%%%%%%%%%%%%%%%%%%%%%%%%%%%%%%%%%
%Texto 

\lipsum[100]

%%%%%%%%%%%%%%%%%%%%%%%%%%%%%%%%%%%%%%%%%%%%%%%%%%%%%%%%%%%%%%%%%%%%%%%%%%%%%%%% 


% ---
% QUARTO SUBCAPÍTULO
% ---
\section{Previsões com modelo ARIMA Sazonal - SARIMA}
% ---


%%%%%%%%%%%%%%%%%%%%%%%%%%%%%%%%%%%%%%%%%%%%%%%%%%%%%%%%%%%%%%%%%%%%%%%%%%%%%%%%
%Texto 

\lipsum[100]

%%%%%%%%%%%%%%%%%%%%%%%%%%%%%%%%%%%%%%%%%%%%%%%%%%%%%%%%%%%%%%%%%%%%%%%%%%%%%%%% 

% ----------------------------------------------------------
% PARTE
% ----------------------------------------------------------
\part{Resultados}
% ----------------------------------------------------------

%---
% P R I M E I R O  C A P Í T U L O 
% ---
\chapter{Análise Exploratória dos dados}
% ---


%%%%%%%%%%%%%%%%%%%%%%%%%%%%%%%%%%%%%%%%%%%%%%%%%%%%%%%%%%%%%%%%%%%%%%%%%%%%%%%%
%Texto 

\lipsum[100]

%%%%%%%%%%%%%%%%%%%%%%%%%%%%%%%%%%%%%%%%%%%%%%%%%%%%%%%%%%%%%%%%%%%%%%%%%%%%%%%%


% ---
% PRIMEIRO SUBCAPÍTULO
% ---
\section{Concentração horária média}
% ---

%%%%%%%%%%%%%%%%%%%%%%%%%%%%%%%%%%%%%%%%%%%%%%%%%%%%%%%%%%%%%%%%%%%%%%%%%%%%%%%%
%Texto 

\lipsum[100]

%%%%%%%%%%%%%%%%%%%%%%%%%%%%%%%%%%%%%%%%%%%%%%%%%%%%%%%%%%%%%%%%%%%%%%%%%%%%%%%%


% ------------------------------------------------------------------------------
%Imagem 1

\begin{figure}[H]
    \centering
    \includegraphics[max size={\textwidth}{\textheight},keepaspectratio=true,
    dpi=500]{./images/graphs/o3_horario_8h.jpeg}
    \caption{Xxxxxxxxxxxxxxxxxxxxxxxxxxxxxxxxxxxxxxxxxxxx.}
    \label{fig:o3_horario_8h.jpeg}
    \caption*{Fonte: Autor, \imprimirdata.}
\end{figure}

% ------------------------------------------------------------------------------


%%%%%%%%%%%%%%%%%%%%%%%%%%%%%%%%%%%%%%%%%%%%%%%%%%%%%%%%%%%%%%%%%%%%%%%%%%%%%%%%
%Texto 

\lipsum[100]

%%%%%%%%%%%%%%%%%%%%%%%%%%%%%%%%%%%%%%%%%%%%%%%%%%%%%%%%%%%%%%%%%%%%%%%%%%%%%%%%


% ------------------------------------------------------------------------------
%Imagem 2

\begin{figure}[H]
    \centering
    \includegraphics[max size={\textwidth}{\textheight},keepaspectratio=true,
    dpi=500]{./images/graphs/no_horario.jpeg}
    \caption{Xxxxxxxxxxxxxxxxxxxxxxxxxxxxxxxxxxxxxxxxxxxx.}
    \label{fig:no_horario.jpeg}
    \caption*{Fonte: Autor, \imprimirdata.}
\end{figure}

% ------------------------------------------------------------------------------


%%%%%%%%%%%%%%%%%%%%%%%%%%%%%%%%%%%%%%%%%%%%%%%%%%%%%%%%%%%%%%%%%%%%%%%%%%%%%%%%
%Texto 

\lipsum[100]

%%%%%%%%%%%%%%%%%%%%%%%%%%%%%%%%%%%%%%%%%%%%%%%%%%%%%%%%%%%%%%%%%%%%%%%%%%%%%%%%


% ------------------------------------------------------------------------------
%Imagem 3

\begin{figure}[H]
    \centering
    \includegraphics[max size={\textwidth}{\textheight},keepaspectratio=true,
    dpi=500]{./images/graphs/no2_horario.jpeg}
    \caption{Xxxxxxxxxxxxxxxxxxxxxxxxxxxxxxxxxxxxxxxxxxxx.}
    \label{fig:no2_horario.jpeg}
    \caption*{Fonte: Autor, \imprimirdata.}
\end{figure}

% ------------------------------------------------------------------------------


%%%%%%%%%%%%%%%%%%%%%%%%%%%%%%%%%%%%%%%%%%%%%%%%%%%%%%%%%%%%%%%%%%%%%%%%%%%%%%%%
%Texto 

\lipsum[100]

%%%%%%%%%%%%%%%%%%%%%%%%%%%%%%%%%%%%%%%%%%%%%%%%%%%%%%%%%%%%%%%%%%%%%%%%%%%%%%%% 


% ---
% SEGUNDO SUBCAPÍTULO
% ---
\section{Perfil de concentração por hora do dia}
% ---


%%%%%%%%%%%%%%%%%%%%%%%%%%%%%%%%%%%%%%%%%%%%%%%%%%%%%%%%%%%%%%%%%%%%%%%%%%%%%%%%
%Texto 

\lipsum[100]

%%%%%%%%%%%%%%%%%%%%%%%%%%%%%%%%%%%%%%%%%%%%%%%%%%%%%%%%%%%%%%%%%%%%%%%%%%%%%%%% 


% ------------------------------------------------------------------------------
%Imagem 4

\begin{figure}[H]
    \centering
    \includegraphics[max size={\textwidth}{\textheight},keepaspectratio=true,
    dpi=500]{./images/graphs/o3_boxplot_hora_do_dia.png}
    \caption{Xxxxxxxxxxxxxxxxxxxxxxxxxxxxxxxxxxxxxxxxxxxx.}
    \label{fig:o3_boxplot_hora_do_dia.png}
    \caption*{Fonte: Autor, \imprimirdata.}
\end{figure}

% ------------------------------------------------------------------------------


%%%%%%%%%%%%%%%%%%%%%%%%%%%%%%%%%%%%%%%%%%%%%%%%%%%%%%%%%%%%%%%%%%%%%%%%%%%%%%%%
%Texto 

\lipsum[100]

%%%%%%%%%%%%%%%%%%%%%%%%%%%%%%%%%%%%%%%%%%%%%%%%%%%%%%%%%%%%%%%%%%%%%%%%%%%%%%%% 


% ------------------------------------------------------------------------------
%Imagem 5

\begin{figure}[H]
    \centering
    \includegraphics[max size={\textwidth}{\textheight},keepaspectratio=true,
    dpi=500]{./images/graphs/no_boxplot_hora_do_dia.png}
    \caption{Xxxxxxxxxxxxxxxxxxxxxxxxxxxxxxxxxxxxxxxxxxxx.}
    \label{fig:no_boxplot_hora_do_dia.png}
    \caption*{Fonte: Autor, \imprimirdata.}
\end{figure}

% ------------------------------------------------------------------------------


%%%%%%%%%%%%%%%%%%%%%%%%%%%%%%%%%%%%%%%%%%%%%%%%%%%%%%%%%%%%%%%%%%%%%%%%%%%%%%%%
%Texto 

\lipsum[100]

%%%%%%%%%%%%%%%%%%%%%%%%%%%%%%%%%%%%%%%%%%%%%%%%%%%%%%%%%%%%%%%%%%%%%%%%%%%%%%%% 


% ------------------------------------------------------------------------------
%Imagem 6

\begin{figure}[H]
    \centering
    \includegraphics[max size={\textwidth}{\textheight},keepaspectratio=true,
    dpi=500]{./images/graphs/no2_boxplot_hora_do_dia.png}
    \caption{Xxxxxxxxxxxxxxxxxxxxxxxxxxxxxxxxxxxxxxxxxxxx.}
    \label{fig:no2_boxplot_hora_do_dia.png}
    \caption*{Fonte: Autor, \imprimirdata.}
\end{figure}

% ------------------------------------------------------------------------------


%%%%%%%%%%%%%%%%%%%%%%%%%%%%%%%%%%%%%%%%%%%%%%%%%%%%%%%%%%%%%%%%%%%%%%%%%%%%%%%%
%Texto 

\lipsum[100]

%%%%%%%%%%%%%%%%%%%%%%%%%%%%%%%%%%%%%%%%%%%%%%%%%%%%%%%%%%%%%%%%%%%%%%%%%%%%%%%% 


% ---
% TERCEIRO SUBCAPÍTULO
% ---
\section{Perfil de concentração por dia da semana}
% ---


%%%%%%%%%%%%%%%%%%%%%%%%%%%%%%%%%%%%%%%%%%%%%%%%%%%%%%%%%%%%%%%%%%%%%%%%%%%%%%%%
%Texto 

\lipsum[100]

%%%%%%%%%%%%%%%%%%%%%%%%%%%%%%%%%%%%%%%%%%%%%%%%%%%%%%%%%%%%%%%%%%%%%%%%%%%%%%%% 


% ------------------------------------------------------------------------------
%Imagem 7

\begin{figure}[H]
    \centering
    \includegraphics[max size={\textwidth}{\textheight},keepaspectratio=true,
    dpi=500]{./images/graphs/o3_boxplot_dia_da_semana.png}
    \caption{Xxxxxxxxxxxxxxxxxxxxxxxxxxxxxxxxxxxxxxxxxxxx.}
    \label{fig:o3_boxplot_dia_da_semana.png}
    \caption*{Fonte: Autor, \imprimirdata.}
\end{figure}

% ------------------------------------------------------------------------------


%%%%%%%%%%%%%%%%%%%%%%%%%%%%%%%%%%%%%%%%%%%%%%%%%%%%%%%%%%%%%%%%%%%%%%%%%%%%%%%%
%Texto 

\lipsum[100]

%%%%%%%%%%%%%%%%%%%%%%%%%%%%%%%%%%%%%%%%%%%%%%%%%%%%%%%%%%%%%%%%%%%%%%%%%%%%%%%% 


% ------------------------------------------------------------------------------
%Imagem 8

\begin{figure}[H]
    \centering
    \includegraphics[max size={\textwidth}{\textheight},keepaspectratio=true,
    dpi=500]{./images/graphs/no_boxplot_dia_da_semana.png}
    \caption{Xxxxxxxxxxxxxxxxxxxxxxxxxxxxxxxxxxxxxxxxxxxx.}
    \label{fig:no_boxplot_dia_da_semana.png}
    \caption*{Fonte: Autor, \imprimirdata.}
\end{figure}

% ------------------------------------------------------------------------------


%%%%%%%%%%%%%%%%%%%%%%%%%%%%%%%%%%%%%%%%%%%%%%%%%%%%%%%%%%%%%%%%%%%%%%%%%%%%%%%%
%Texto 

\lipsum[100]

%%%%%%%%%%%%%%%%%%%%%%%%%%%%%%%%%%%%%%%%%%%%%%%%%%%%%%%%%%%%%%%%%%%%%%%%%%%%%%%% 


% ------------------------------------------------------------------------------
%Imagem 9

\begin{figure}[H]
    \centering
    \includegraphics[max size={\textwidth}{\textheight},keepaspectratio=true,
    dpi=500]{./images/graphs/no2_boxplot_dia_da_semana.png}
    \caption{Xxxxxxxxxxxxxxxxxxxxxxxxxxxxxxxxxxxxxxxxxxxx.}
    \label{fig:no2_boxplot_dia_da_semana.png}
    \caption*{Fonte: Autor, \imprimirdata.}
\end{figure}

% ------------------------------------------------------------------------------


%%%%%%%%%%%%%%%%%%%%%%%%%%%%%%%%%%%%%%%%%%%%%%%%%%%%%%%%%%%%%%%%%%%%%%%%%%%%%%%%
%Texto 

\lipsum[100]

%%%%%%%%%%%%%%%%%%%%%%%%%%%%%%%%%%%%%%%%%%%%%%%%%%%%%%%%%%%%%%%%%%%%%%%%%%%%%%%% 


% ---
% QUARTO SUBCAPÍTULO
% ---
\section{Perfil de concentração por mês do ano}
% ---


%%%%%%%%%%%%%%%%%%%%%%%%%%%%%%%%%%%%%%%%%%%%%%%%%%%%%%%%%%%%%%%%%%%%%%%%%%%%%%%%
%Texto 

\lipsum[100]

%%%%%%%%%%%%%%%%%%%%%%%%%%%%%%%%%%%%%%%%%%%%%%%%%%%%%%%%%%%%%%%%%%%%%%%%%%%%%%%% 


% ------------------------------------------------------------------------------
%Imagem 10

\begin{figure}[H]
    \centering
    \includegraphics[max size={\textwidth}{\textheight},keepaspectratio=true,
    dpi=500]{./images/graphs/o3_boxplot_meses_do_ano.png}
    \caption{Xxxxxxxxxxxxxxxxxxxxxxxxxxxxxxxxxxxxxxxxxxxx.}
    \label{fig:o3_boxplot_meses_do_ano.png}
    \caption*{Fonte: Autor, \imprimirdata.}
\end{figure}

% ------------------------------------------------------------------------------


%%%%%%%%%%%%%%%%%%%%%%%%%%%%%%%%%%%%%%%%%%%%%%%%%%%%%%%%%%%%%%%%%%%%%%%%%%%%%%%%
%Texto 

\lipsum[100]

%%%%%%%%%%%%%%%%%%%%%%%%%%%%%%%%%%%%%%%%%%%%%%%%%%%%%%%%%%%%%%%%%%%%%%%%%%%%%%%% 


% ------------------------------------------------------------------------------
%Imagem 11

\begin{figure}[H]
    \centering
    \includegraphics[max size={\textwidth}{\textheight},keepaspectratio=true,
    dpi=500]{./images/graphs/no_boxplot_meses_do_ano.png}
    \caption{Xxxxxxxxxxxxxxxxxxxxxxxxxxxxxxxxxxxxxxxxxxxx.}
    \label{fig:no_boxplot_meses_do_ano.png}
    \caption*{Fonte: Autor, \imprimirdata.}
\end{figure}

% ------------------------------------------------------------------------------


%%%%%%%%%%%%%%%%%%%%%%%%%%%%%%%%%%%%%%%%%%%%%%%%%%%%%%%%%%%%%%%%%%%%%%%%%%%%%%%%
%Texto 

\lipsum[100]

%%%%%%%%%%%%%%%%%%%%%%%%%%%%%%%%%%%%%%%%%%%%%%%%%%%%%%%%%%%%%%%%%%%%%%%%%%%%%%%% 


% ------------------------------------------------------------------------------
%Imagem 12

\begin{figure}[H]
    \centering
    \includegraphics[max size={\textwidth}{\textheight},keepaspectratio=true,
    dpi=500]{./images/graphs/no2_boxplot_meses_do_ano.png}
    \caption{Xxxxxxxxxxxxxxxxxxxxxxxxxxxxxxxxxxxxxxxxxxxx.}
    \label{fig:no2_boxplot_meses_do_ano.png}
    \caption*{Fonte: Autor, \imprimirdata.}
\end{figure}

% ------------------------------------------------------------------------------


%%%%%%%%%%%%%%%%%%%%%%%%%%%%%%%%%%%%%%%%%%%%%%%%%%%%%%%%%%%%%%%%%%%%%%%%%%%%%%%%
%Texto 

\lipsum[100]

%%%%%%%%%%%%%%%%%%%%%%%%%%%%%%%%%%%%%%%%%%%%%%%%%%%%%%%%%%%%%%%%%%%%%%%%%%%%%%%% 


% ---
% QUINTO SUBCAPÍTULO
% ---
\section{Perfil de concentração por estação do ano}
% ---


%%%%%%%%%%%%%%%%%%%%%%%%%%%%%%%%%%%%%%%%%%%%%%%%%%%%%%%%%%%%%%%%%%%%%%%%%%%%%%%%
%Texto 

\lipsum[100]

%%%%%%%%%%%%%%%%%%%%%%%%%%%%%%%%%%%%%%%%%%%%%%%%%%%%%%%%%%%%%%%%%%%%%%%%%%%%%%%% 


% ------------------------------------------------------------------------------
%Imagem 13

\begin{figure}[H]
    \centering
    \includegraphics[max size={\textwidth}{\textheight},keepaspectratio=true,
    dpi=500]{./images/graphs/o3_boxplot_estacoes_do_ano.png}
    \caption{Xxxxxxxxxxxxxxxxxxxxxxxxxxxxxxxxxxxxxxxxxxxx.}
    \label{fig:o3_boxplot_estacoes_do_ano.png}
    \caption*{Fonte: Autor, \imprimirdata.}
\end{figure}

% ------------------------------------------------------------------------------


%%%%%%%%%%%%%%%%%%%%%%%%%%%%%%%%%%%%%%%%%%%%%%%%%%%%%%%%%%%%%%%%%%%%%%%%%%%%%%%%
%Texto 

\lipsum[100]

%%%%%%%%%%%%%%%%%%%%%%%%%%%%%%%%%%%%%%%%%%%%%%%%%%%%%%%%%%%%%%%%%%%%%%%%%%%%%%%% 


% ------------------------------------------------------------------------------
%Imagem 14

\begin{figure}[H]
    \centering
    \includegraphics[max size={\textwidth}{\textheight},keepaspectratio=true,
    dpi=500]{./images/graphs/no_boxplot_estacoes_do_ano.png}
    \caption{Xxxxxxxxxxxxxxxxxxxxxxxxxxxxxxxxxxxxxxxxxxxx.}
    \label{fig:no_boxplot_estacoes_do_ano.png}
    \caption*{Fonte: Autor, \imprimirdata.}
\end{figure}

% ------------------------------------------------------------------------------


%%%%%%%%%%%%%%%%%%%%%%%%%%%%%%%%%%%%%%%%%%%%%%%%%%%%%%%%%%%%%%%%%%%%%%%%%%%%%%%%
%Texto 

\lipsum[100]

%%%%%%%%%%%%%%%%%%%%%%%%%%%%%%%%%%%%%%%%%%%%%%%%%%%%%%%%%%%%%%%%%%%%%%%%%%%%%%%% 


% ------------------------------------------------------------------------------
%Imagem 15

\begin{figure}[H]
    \centering
    \includegraphics[max size={\textwidth}{\textheight},keepaspectratio=true,
    dpi=500]{./images/graphs/no2_boxplot_estacoes_do_ano.png}
    \caption{Xxxxxxxxxxxxxxxxxxxxxxxxxxxxxxxxxxxxxxxxxxxx.}
    \label{fig:no2_boxplot_estacoes_do_ano.png}
    \caption*{Fonte: Autor, \imprimirdata.}
\end{figure}

% ------------------------------------------------------------------------------


%%%%%%%%%%%%%%%%%%%%%%%%%%%%%%%%%%%%%%%%%%%%%%%%%%%%%%%%%%%%%%%%%%%%%%%%%%%%%%%%
%Texto 

\lipsum[100]

%%%%%%%%%%%%%%%%%%%%%%%%%%%%%%%%%%%%%%%%%%%%%%%%%%%%%%%%%%%%%%%%%%%%%%%%%%%%%%%% 


% ---
% SEXTO SUBCAPÍTULO
% ---
\section{Dados meteorológicos}
% ---


%%%%%%%%%%%%%%%%%%%%%%%%%%%%%%%%%%%%%%%%%%%%%%%%%%%%%%%%%%%%%%%%%%%%%%%%%%%%%%%%
%Texto 

\lipsum[100]

%%%%%%%%%%%%%%%%%%%%%%%%%%%%%%%%%%%%%%%%%%%%%%%%%%%%%%%%%%%%%%%%%%%%%%%%%%%%%%%% 


% ------------------------------------------------------------------------------
%Imagem 16

\begin{figure}[H]
    \centering
    \includegraphics[max size={\textwidth}{\textheight},keepaspectratio=true,
    dpi=500]{./images/graphs/meteorologia_dia.jpg}
    \caption{Xxxxxxxxxxxxxxxxxxxxxxxxxxxxxxxxxxxxxxxxxxxx.}
    \label{fig:meteorologia_dia.jpg}
    \caption*{Fonte: Autor, \imprimirdata.}
\end{figure}

% ------------------------------------------------------------------------------


%%%%%%%%%%%%%%%%%%%%%%%%%%%%%%%%%%%%%%%%%%%%%%%%%%%%%%%%%%%%%%%%%%%%%%%%%%%%%%%%
%Texto 

\lipsum[100]

%%%%%%%%%%%%%%%%%%%%%%%%%%%%%%%%%%%%%%%%%%%%%%%%%%%%%%%%%%%%%%%%%%%%%%%%%%%%%%%% 


% ------------------------------------------------------------------------------
%Imagem 17

\begin{figure}[H]
    \centering
    \includegraphics[max size={\textwidth}{\textheight},keepaspectratio=true,
    dpi=500]{./images/graphs/meteorologia_mes.jpg}
    \caption{Xxxxxxxxxxxxxxxxxxxxxxxxxxxxxxxxxxxxxxxxxxxx.}
    \label{fig:meteorologia_mes.jpg}
    \caption*{Fonte: Autor, \imprimirdata.}
\end{figure}

% ------------------------------------------------------------------------------


%%%%%%%%%%%%%%%%%%%%%%%%%%%%%%%%%%%%%%%%%%%%%%%%%%%%%%%%%%%%%%%%%%%%%%%%%%%%%%%%
%Texto 

\lipsum[100]

%%%%%%%%%%%%%%%%%%%%%%%%%%%%%%%%%%%%%%%%%%%%%%%%%%%%%%%%%%%%%%%%%%%%%%%%%%%%%%%% 


%---
% S E G U N D O  C A P Í T U L O 
% ---
\chapter{Regressão Linear}
% ---


%%%%%%%%%%%%%%%%%%%%%%%%%%%%%%%%%%%%%%%%%%%%%%%%%%%%%%%%%%%%%%%%%%%%%%%%%%%%%%%%
%Texto 

\lipsum[100]

%%%%%%%%%%%%%%%%%%%%%%%%%%%%%%%%%%%%%%%%%%%%%%%%%%%%%%%%%%%%%%%%%%%%%%%%%%%%%%%% 


% ------------------------------------------------------------------------------
%Imagem 18

\begin{figure}[H]
    \centering
    \includegraphics[max size={\textwidth}{\textheight},keepaspectratio=true,
    dpi=500]{./images/graphs/lin_regr_8h_dia.png}
    \caption{Xxxxxxxxxxxxxxxxxxxxxxxxxxxxxxxxxxxxxxxxxxxx.}
    \label{fig:lin_regr_8h_dia.png}
    \caption*{Fonte: Autor, \imprimirdata.}
\end{figure}

% ------------------------------------------------------------------------------


%%%%%%%%%%%%%%%%%%%%%%%%%%%%%%%%%%%%%%%%%%%%%%%%%%%%%%%%%%%%%%%%%%%%%%%%%%%%%%%%
%Texto 

\lipsum[100]

%%%%%%%%%%%%%%%%%%%%%%%%%%%%%%%%%%%%%%%%%%%%%%%%%%%%%%%%%%%%%%%%%%%%%%%%%%%%%%%% 


% ------------------------------------------------------------------------------
%Imagem 19

\begin{figure}[H]
    \centering
    \includegraphics[max size={\textwidth}{\textheight},keepaspectratio=true,
    dpi=500]{./images/graphs/regr_lin_hora_cetesb.png}
    \caption{Xxxxxxxxxxxxxxxxxxxxxxxxxxxxxxxxxxxxxxxxxxxx.}
    \label{fig:regr_lin_hora_cetesb.png}
    \caption*{Fonte: Autor, \imprimirdata.}
\end{figure}

% ------------------------------------------------------------------------------


%%%%%%%%%%%%%%%%%%%%%%%%%%%%%%%%%%%%%%%%%%%%%%%%%%%%%%%%%%%%%%%%%%%%%%%%%%%%%%%%
%Texto 

\lipsum[100]

%%%%%%%%%%%%%%%%%%%%%%%%%%%%%%%%%%%%%%%%%%%%%%%%%%%%%%%%%%%%%%%%%%%%%%%%%%%%%%%% 


% ---
% PRIMEIRO SUBCAPÍTULO
% ---
\section{Mapas de calor}
% ---


%%%%%%%%%%%%%%%%%%%%%%%%%%%%%%%%%%%%%%%%%%%%%%%%%%%%%%%%%%%%%%%%%%%%%%%%%%%%%%%%
%Texto 

\lipsum[100]

%%%%%%%%%%%%%%%%%%%%%%%%%%%%%%%%%%%%%%%%%%%%%%%%%%%%%%%%%%%%%%%%%%%%%%%%%%%%%%%% 


% ------------------------------------------------------------------------------
%Imagem 20

\begin{figure}[H]
    \centering
    \includegraphics[max size={\textwidth}{\textheight},keepaspectratio=true,
    dpi=500]{./images/graphs/coef_pearson_horario.png}
    \caption{Xxxxxxxxxxxxxxxxxxxxxxxxxxxxxxxxxxxxxxxxxxxx.}
    \label{fig:coef_pearson_horario.png}
    \caption*{Fonte: Autor, \imprimirdata.}
\end{figure}

% ------------------------------------------------------------------------------


%%%%%%%%%%%%%%%%%%%%%%%%%%%%%%%%%%%%%%%%%%%%%%%%%%%%%%%%%%%%%%%%%%%%%%%%%%%%%%%%
%Texto 

\lipsum[100]

%%%%%%%%%%%%%%%%%%%%%%%%%%%%%%%%%%%%%%%%%%%%%%%%%%%%%%%%%%%%%%%%%%%%%%%%%%%%%%%% 


%---
% T E R C E I R O  C A P Í T U L O 
% ---
\chapter{Decomposição das Séries Temporais}
% ---

% ---
% PRIMEIRO SUBCAPÍTULO
% ---
\section{Sazonalidade}
% ---


%%%%%%%%%%%%%%%%%%%%%%%%%%%%%%%%%%%%%%%%%%%%%%%%%%%%%%%%%%%%%%%%%%%%%%%%%%%%%%%%
%Texto 

\lipsum[100]

%%%%%%%%%%%%%%%%%%%%%%%%%%%%%%%%%%%%%%%%%%%%%%%%%%%%%%%%%%%%%%%%%%%%%%%%%%%%%%%% 


% ------------------------------------------------------------------------------
%Imagem 21

\begin{figure}[H]
    \centering
    \includegraphics[max size={\textwidth}{\textheight},keepaspectratio=true,
    dpi=500]{./images/graphs/cp_ets_seasonality.jpg}
    \caption{Xxxxxxxxxxxxxxxxxxxxxxxxxxxxxxxxxxxxxxxxxxxx.}
    \label{fig:cp_ets_seasonality.jpg}
    \caption*{Fonte: Autor, \imprimirdata.}
\end{figure}

% ------------------------------------------------------------------------------


%%%%%%%%%%%%%%%%%%%%%%%%%%%%%%%%%%%%%%%%%%%%%%%%%%%%%%%%%%%%%%%%%%%%%%%%%%%%%%%%
%Texto 

\lipsum[100]

%%%%%%%%%%%%%%%%%%%%%%%%%%%%%%%%%%%%%%%%%%%%%%%%%%%%%%%%%%%%%%%%%%%%%%%%%%%%%%%% 


% ------------------------------------------------------------------------------
%Imagem 22

\begin{figure}[H]
    \centering
    \includegraphics[max size={\textwidth}{\textheight},keepaspectratio=true,
    dpi=500]{./images/graphs/ib_ets_seasonality.jpg}
    \caption{Xxxxxxxxxxxxxxxxxxxxxxxxxxxxxxxxxxxxxxxxxxxx.}
    \label{fig:ib_ets_seasonality.jpg}
    \caption*{Fonte: Autor, \imprimirdata.}
\end{figure}

% ------------------------------------------------------------------------------


%%%%%%%%%%%%%%%%%%%%%%%%%%%%%%%%%%%%%%%%%%%%%%%%%%%%%%%%%%%%%%%%%%%%%%%%%%%%%%%%
%Texto 

\lipsum[100]

%%%%%%%%%%%%%%%%%%%%%%%%%%%%%%%%%%%%%%%%%%%%%%%%%%%%%%%%%%%%%%%%%%%%%%%%%%%%%%%% 



% ------------------------------------------------------------------------------
%Imagem 23

\begin{figure}[H]
    \centering
    \includegraphics[max size={\textwidth}{\textheight},keepaspectratio=true,
    dpi=500]{./images/graphs/inter_ets_seasonality.jpg}
    \caption{Xxxxxxxxxxxxxxxxxxxxxxxxxxxxxxxxxxxxxxxxxxxx.}
    \label{fig:inter_ets_seasonality.jpg}
    \caption*{Fonte: Autor, \imprimirdata.}
\end{figure}

% ------------------------------------------------------------------------------


%%%%%%%%%%%%%%%%%%%%%%%%%%%%%%%%%%%%%%%%%%%%%%%%%%%%%%%%%%%%%%%%%%%%%%%%%%%%%%%%
%Texto 

\lipsum[100]

%%%%%%%%%%%%%%%%%%%%%%%%%%%%%%%%%%%%%%%%%%%%%%%%%%%%%%%%%%%%%%%%%%%%%%%%%%%%%%%% 


% ------------------------------------------------------------------------------
%Imagem 24

\begin{figure}[H]
    \centering
    \includegraphics[max size={\textwidth}{\textheight},keepaspectratio=true,
    dpi=500]{./images/graphs/pq_ets_seasonality.jpg}
    \caption{Xxxxxxxxxxxxxxxxxxxxxxxxxxxxxxxxxxxxxxxxxxxx.}
    \label{fig:pq_ets_seasonality.jpg}
    \caption*{Fonte: Autor, \imprimirdata.}
\end{figure}

% ------------------------------------------------------------------------------


%%%%%%%%%%%%%%%%%%%%%%%%%%%%%%%%%%%%%%%%%%%%%%%%%%%%%%%%%%%%%%%%%%%%%%%%%%%%%%%%
%Texto 

\lipsum[100]

%%%%%%%%%%%%%%%%%%%%%%%%%%%%%%%%%%%%%%%%%%%%%%%%%%%%%%%%%%%%%%%%%%%%%%%%%%%%%%%% 


% ---
% SEGUNDO SUBCAPÍTULO
% ---
\section{Tendência}
% ---


%%%%%%%%%%%%%%%%%%%%%%%%%%%%%%%%%%%%%%%%%%%%%%%%%%%%%%%%%%%%%%%%%%%%%%%%%%%%%%%%
%Texto 

\lipsum[100]

%%%%%%%%%%%%%%%%%%%%%%%%%%%%%%%%%%%%%%%%%%%%%%%%%%%%%%%%%%%%%%%%%%%%%%%%%%%%%%%% 


% ------------------------------------------------------------------------------
%Imagem 25

\begin{figure}[H]
    \centering
    \includegraphics[max size={\textwidth}{\textheight},keepaspectratio=true,
    dpi=500]{./images/graphs/cp_ets_trend.jpg}
    \caption{Xxxxxxxxxxxxxxxxxxxxxxxxxxxxxxxxxxxxxxxxxxxx.}
    \label{fig:cp_ets_trend.jpg}
    \caption*{Fonte: Autor, \imprimirdata.}
\end{figure}

% ------------------------------------------------------------------------------


%%%%%%%%%%%%%%%%%%%%%%%%%%%%%%%%%%%%%%%%%%%%%%%%%%%%%%%%%%%%%%%%%%%%%%%%%%%%%%%%
%Texto 

\lipsum[100]

%%%%%%%%%%%%%%%%%%%%%%%%%%%%%%%%%%%%%%%%%%%%%%%%%%%%%%%%%%%%%%%%%%%%%%%%%%%%%%%% 


% ------------------------------------------------------------------------------
%Imagem 26

\begin{figure}[H]
    \centering
    \includegraphics[max size={\textwidth}{\textheight},keepaspectratio=true,
    dpi=500]{./images/graphs/ib_ets_trend.jpg}
    \caption{Xxxxxxxxxxxxxxxxxxxxxxxxxxxxxxxxxxxxxxxxxxxx.}
    \label{fig:ib_ets_trend.jpg}
    \caption*{Fonte: Autor, \imprimirdata.}
\end{figure}

% ------------------------------------------------------------------------------


%%%%%%%%%%%%%%%%%%%%%%%%%%%%%%%%%%%%%%%%%%%%%%%%%%%%%%%%%%%%%%%%%%%%%%%%%%%%%%%%
%Texto 

\lipsum[100]

%%%%%%%%%%%%%%%%%%%%%%%%%%%%%%%%%%%%%%%%%%%%%%%%%%%%%%%%%%%%%%%%%%%%%%%%%%%%%%%% 


% ------------------------------------------------------------------------------
%Imagem 27

\begin{figure}[H]
    \centering
    \includegraphics[max size={\textwidth}{\textheight},keepaspectratio=true,
    dpi=500]{./images/graphs/inter_ets_trend.jpg}
    \caption{Xxxxxxxxxxxxxxxxxxxxxxxxxxxxxxxxxxxxxxxxxxxx.}
    \label{fig:inter_ets_trend.jpg}
    \caption*{Fonte: Autor, \imprimirdata.}
\end{figure}

% ------------------------------------------------------------------------------


%%%%%%%%%%%%%%%%%%%%%%%%%%%%%%%%%%%%%%%%%%%%%%%%%%%%%%%%%%%%%%%%%%%%%%%%%%%%%%%%
%Texto 

\lipsum[100]

%%%%%%%%%%%%%%%%%%%%%%%%%%%%%%%%%%%%%%%%%%%%%%%%%%%%%%%%%%%%%%%%%%%%%%%%%%%%%%%% 


% ------------------------------------------------------------------------------
%Imagem 28

\begin{figure}[H]
    \centering
    \includegraphics[max size={\textwidth}{\textheight},keepaspectratio=true,
    dpi=500]{./images/graphs/pq_ets_trend.jpg}
    \caption{Xxxxxxxxxxxxxxxxxxxxxxxxxxxxxxxxxxxxxxxxxxxx.}
    \label{fig:pq_ets_trend.jpg}
    \caption*{Fonte: Autor, \imprimirdata.}
\end{figure}

% ------------------------------------------------------------------------------


%%%%%%%%%%%%%%%%%%%%%%%%%%%%%%%%%%%%%%%%%%%%%%%%%%%%%%%%%%%%%%%%%%%%%%%%%%%%%%%%
%Texto 

\lipsum[100]

%%%%%%%%%%%%%%%%%%%%%%%%%%%%%%%%%%%%%%%%%%%%%%%%%%%%%%%%%%%%%%%%%%%%%%%%%%%%%%%% 


%---
% Q U A R T O  C A P Í T U L O 
% ---
\chapter{SARIMA}
% ---


%%%%%%%%%%%%%%%%%%%%%%%%%%%%%%%%%%%%%%%%%%%%%%%%%%%%%%%%%%%%%%%%%%%%%%%%%%%%%%%%
%Texto 

\lipsum[100]

%%%%%%%%%%%%%%%%%%%%%%%%%%%%%%%%%%%%%%%%%%%%%%%%%%%%%%%%%%%%%%%%%%%%%%%%%%%%%%%% 


% ---
% PRIMEIRO SUBCAPÍTULO
% ---
\section{Algoritmo de \textit{Machine Learning}}
% ---


%%%%%%%%%%%%%%%%%%%%%%%%%%%%%%%%%%%%%%%%%%%%%%%%%%%%%%%%%%%%%%%%%%%%%%%%%%%%%%%%
%Texto 

\lipsum[100]

%%%%%%%%%%%%%%%%%%%%%%%%%%%%%%%%%%%%%%%%%%%%%%%%%%%%%%%%%%%%%%%%%%%%%%%%%%%%%%%% 


% ---
% SEGUNDO SUBCAPÍTULO
% ---
\section{Validação do algoritmo}
% ---


%%%%%%%%%%%%%%%%%%%%%%%%%%%%%%%%%%%%%%%%%%%%%%%%%%%%%%%%%%%%%%%%%%%%%%%%%%%%%%%%
%Texto 

\lipsum[100]

%%%%%%%%%%%%%%%%%%%%%%%%%%%%%%%%%%%%%%%%%%%%%%%%%%%%%%%%%%%%%%%%%%%%%%%%%%%%%%%% 


% ---
% TERCEIRO SUBCAPÍTULO
% ---
\section{Previsões}
% ---


%%%%%%%%%%%%%%%%%%%%%%%%%%%%%%%%%%%%%%%%%%%%%%%%%%%%%%%%%%%%%%%%%%%%%%%%%%%%%%%%
%Texto 

\lipsum[100]

%%%%%%%%%%%%%%%%%%%%%%%%%%%%%%%%%%%%%%%%%%%%%%%%%%%%%%%%%%%%%%%%%%%%%%%%%%%%%%%% 


% ------------------------------------------------------------------------------
%Imagem 29

\begin{figure}[H]
    \centering
    \includegraphics[max size={\textwidth}{\textheight},keepaspectratio=true,
    dpi=500]{./images/graphs/cp_prediction.jpg}
    \caption{Xxxxxxxxxxxxxxxxxxxxxxxxxxxxxxxxxxxxxxxxxxxx.}
    \label{fig:cp_prediction.jpg}
    \caption*{Fonte: Autor, \imprimirdata.}
\end{figure}

% ------------------------------------------------------------------------------


%%%%%%%%%%%%%%%%%%%%%%%%%%%%%%%%%%%%%%%%%%%%%%%%%%%%%%%%%%%%%%%%%%%%%%%%%%%%%%%%
%Texto 

\lipsum[100]

%%%%%%%%%%%%%%%%%%%%%%%%%%%%%%%%%%%%%%%%%%%%%%%%%%%%%%%%%%%%%%%%%%%%%%%%%%%%%%%% 


% ------------------------------------------------------------------------------
%Imagem 30

\begin{figure}[H]
    \centering
    \includegraphics[max size={\textwidth}{\textheight},keepaspectratio=true,
    dpi=500]{./images/graphs/ib_prediction.jpg}
    \caption{Xxxxxxxxxxxxxxxxxxxxxxxxxxxxxxxxxxxxxxxxxxxx.}
    \label{fig:ib_prediction.jpg}
    \caption*{Fonte: Autor, \imprimirdata.}
\end{figure}

% ------------------------------------------------------------------------------


%%%%%%%%%%%%%%%%%%%%%%%%%%%%%%%%%%%%%%%%%%%%%%%%%%%%%%%%%%%%%%%%%%%%%%%%%%%%%%%%
%Texto 

\lipsum[100]

%%%%%%%%%%%%%%%%%%%%%%%%%%%%%%%%%%%%%%%%%%%%%%%%%%%%%%%%%%%%%%%%%%%%%%%%%%%%%%%% 


% ------------------------------------------------------------------------------
%Imagem 31

\begin{figure}[H]
    \centering
    \includegraphics[max size={\textwidth}{\textheight},keepaspectratio=true,
    dpi=500]{./images/graphs/inter_prediction.jpg}
    \caption{Xxxxxxxxxxxxxxxxxxxxxxxxxxxxxxxxxxxxxxxxxxxx.}
    \label{fig:inter_prediction.jpg}
    \caption*{Fonte: Autor, \imprimirdata.}
\end{figure}

% ------------------------------------------------------------------------------


%%%%%%%%%%%%%%%%%%%%%%%%%%%%%%%%%%%%%%%%%%%%%%%%%%%%%%%%%%%%%%%%%%%%%%%%%%%%%%%%
%Texto 

\lipsum[100]

%%%%%%%%%%%%%%%%%%%%%%%%%%%%%%%%%%%%%%%%%%%%%%%%%%%%%%%%%%%%%%%%%%%%%%%%%%%%%%%% 


% ------------------------------------------------------------------------------
%Imagem 32

\begin{figure}[H]
    \centering
    \includegraphics[max size={\textwidth}{\textheight},keepaspectratio=true,
    dpi=500]{./images/graphs/pq_prediction.jpg}
    \caption{Xxxxxxxxxxxxxxxxxxxxxxxxxxxxxxxxxxxxxxxxxxxx.}
    \label{fig:pq_prediction.jpg}
    \caption*{Fonte: Autor, \imprimirdata.}
\end{figure}

% ------------------------------------------------------------------------------


%%%%%%%%%%%%%%%%%%%%%%%%%%%%%%%%%%%%%%%%%%%%%%%%%%%%%%%%%%%%%%%%%%%%%%%%%%%%%%%%
%Texto 

\lipsum[100]

%%%%%%%%%%%%%%%%%%%%%%%%%%%%%%%%%%%%%%%%%%%%%%%%%%%%%%%%%%%%%%%%%%%%%%%%%%%%%%%% 

% ----------------------------------------------------------
% Finaliza a parte no bookmark do PDF
% para que se inicie o bookmark na raiz
% e adiciona espaço de parte no Sumário
% ----------------------------------------------------------
\phantompart

% ---
% Conclusão
% ---

% ---
% Conclusão
% ---
\chapter{Conclusão}
% ---

\lipsum[31-33]

%%%%%%%%%%%%%%%%%%%%%%%%%%%%%%%%%%%%%%%%%%%%%%%%%%%%%%%%%%%%%%


%  E L E M E N T O S   P Ó S - T E X T U A I S  (OBRIGATÓRIO)


%%%%%%%%%%%%%%%%%%%%%%%%%%%%%%%%%%%%%%%%%%%%%%%%%%%%%%%%%%%%%%
\postextual
% ----------------------------------------------------------

%%%%%%%%%%%%%%%%%%%%%%%%%%%%%%%%%%%%%%%%%%%%%%%%%%%%%%%%%%%%%
% R E F E R Ê N C I A S  (OBRIGATÓRIO)
%%%%%%%%%%%%%%%%%%%%%%%%%%%%%%%%%%%%%%%%%%%%%%%%%%%%%%%%%%%%%
\bibliography{./content/ref_1,./content/ref_2}

%%%%%%%%%%%%%%%%%%%%%%%%%%%%%%%%%%%%%%%%%%%%%%%%%%%%
% G L O S S A R I O (OPCIONAL)
%%%%%%%%%%%%%%%%%%%%%%%%%%%%%%%%%%%%%%%%%%%%%%%%%%%%
%
% Consulte o manual da classe abntex2 para orientações sobre o glossário.
%
%\glossary

%%%%%%%%%%%%%%%%%%%%%%%%%%%%%%%%%%%%%%%%%%%%%%%%%%%%%%%%%
% A P Ê N D I C E S  (OPCIONALl) 
%%%%%%%%%%%%%%%%%%%%%%%%%%%%%%%%%%%%%%%%%%%%%%%%%%%%%%%%%

% ---
% Inicia os apêndices
% ---
% \begin{apendicesenv}

% % Imprime uma página indicando o início dos apêndices
% \partapendices

% % ----------------------------------------------------------
% \chapter{Quisque libero justo}
% % ----------------------------------------------------------

% \lipsum[50]

% % ----------------------------------------------------------
% \chapter{Nullam elementum urna vel imperdiet sodales elit ipsum pharetra ligula
% ac pretium ante justo a nulla curabitur tristique arcu eu metus}
% % ----------------------------------------------------------
% \lipsum[55-57]

% \end{apendicesenv}
% % ---


%%%%%%%%%%%%%%%%%%%%%%%%%%%%%%%%%%%%%%%%%%%%%%%%%%%%%%%%%
% A N E X O S  (OPCIONAL) 
%%%%%%%%%%%%%%%%%%%%%%%%%%%%%%%%%%%%%%%%%%%%%%%%%%%%%%%%%

% % ---
% % Inicia os anexos
% % ---
% \begin{anexosenv}

% % Imprime uma página indicando o início dos anexos
% \partanexos

% % ---
% \chapter{Morbi ultrices rutrum lorem.}
% % ---
% \lipsum[30]

% % ---
% \chapter{Cras non urna sed feugiat cum sociis natoque penatibus et magnis dis
% parturient montes nascetur ridiculus mus}
% % ---

% \lipsum[31]

% % ---
% \chapter{Fusce facilisis lacinia dui}
% % ---

% \lipsum[32]

% \end{anexosenv}

%%%%%%%%%%%%%%%%%%%%%%%%%%%%%%%%%%%%%%%%%%%%%%%%%%%%%%%%%
% Í N D I C E  R E M I S S I V O  (OPCIONAL) 
%%%%%%%%%%%%%%%%%%%%%%%%%%%%%%%%%%%%%%%%%%%%%%%%%%%%%%%%%
% \phantompart
% \printindex
%---------------------------------------------------------------------

\end{document}
